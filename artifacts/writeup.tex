\documentclass[twocolumn]{article}
\usepackage{graphicx}
\usepackage{float}
\usepackage{lipsum}

\title{Lab 0: Sorting Algorithm Optimization}
\author{Team 22}
\date{September 2024}

\begin{document}

\maketitle

\section{Overview}
This lab focuses on optimizing sorting algorithms for performance, a crucial skill in computer science. We'll start with unoptimized baseline implementations and iteratively create multiple variants to improve efficiency. The main objectives include evaluating different sorting algorithms (with at least two recursive sorts), fine-tuning a specific sort, implementing a dispatch routine to select the optimal sort based on problem size, incorporating dispatch into a recursive sort, and demonstrating low-level optimizations such as instruction-level parallelism. This lab emphasizes hands-on experience in performance optimization, encouraging us to test our hypotheses and refine our approach through multiple iterations. It's an excellent opportunity to apply theoretical knowledge from our coursework to practical problems, bridging the gap between classroom learning and real-world software optimization challenges. The competitive aspect, with bonus points for top performers, adds an exciting dimension to the project and motivates us to push our optimization skills to the limit.

\section{Course Relevance}
This lab assignment integrates concepts from several courses in our computer science curriculum. Our \textit{Data Structures and Algorithms} course laid the foundation for understanding various sorting algorithms and their complexity, which is crucial for this optimization task. \textit{Computer Architecture} provides insights into hardware-level optimizations, especially relevant for tasks involving instruction-level parallelism and cache utilization. The \textit{Operating Systems} course helps us understand process management and memory hierarchies, essential for efficient sorting implementations. Our \textit{Programming Languages} course aids in writing efficient code across different paradigms, including recursive implementations. The \textit{Software Engineering} course's emphasis on iterative development and performance analysis aligns perfectly with the lab's approach to creating and evaluating multiple variants of sorting algorithms. Lastly, our \textit{Analysis of Algorithms} course provides the theoretical backdrop for understanding time complexity and space-time tradeoffs, which is vital when comparing and optimizing different sorting techniques.

\section{Refinements}

\subsection{Variant 1: Optimized Quicksort}

\subsubsection{Changes Made}
In this variant, we implemented an optimized version of the Quicksort algorithm. The main changes include:
\begin{itemize}
    \item Using a median-of-three method for pivot selection
    \item Implementing an insertion sort for small subarrays (n < 10)
    \item Utilizing tail recursion elimination
\end{itemize}

\subsubsection{Expected Results}
We expect to see improved performance, especially for larger datasets. The median-of-three pivot selection should provide better partitioning, reducing the likelihood of worst-case scenarios. The insertion sort for small subarrays should decrease the overhead of recursion for small inputs, and tail recursion elimination should help in reducing stack usage.

\subsubsection{Performance Analysis}
%\begin{figure}[H]
%    \centering
%    \includegraphics[width=0.8\columnwidth]{placeholder_plot1.png}
%    \caption{Performance plot for Optimized Quicksort}
%    \label{fig:variant1_plot}
%\end{figure}

The performance plot shows a significant improvement in execution time compared to the baseline Quicksort implementation. We observe that:
\begin{itemize}
    \item For small input sizes (n < 1000), the optimized version performs slightly better due to the insertion sort optimization.
    \item For larger inputs, we see a more pronounced improvement, with the optimized version consistently outperforming the baseline.
    \item The performance gain increases with input size, suggesting that our optimizations are particularly effective for larger datasets.
\end{itemize}

\subsubsection{Potential Improvements}
To further enhance this variant, we could:
\begin{itemize}
    \item Experiment with different thresholds for switching to insertion sort
    \item Implement a three-way partitioning scheme to handle duplicate elements more efficiently
    \item Explore cache-friendly memory access patterns to improve performance on modern hardware
\end{itemize}

\subsection{Variant 2: Hybrid Merge-Insertion Sort}

\subsubsection{Changes Made}
For this variant, we created a hybrid algorithm combining Merge Sort and Insertion Sort:
\begin{itemize}
    \item Used Merge Sort as the primary algorithm
    \item Switched to Insertion Sort for small subarrays (n < 64)
    \item Implemented an in-place merge function to reduce memory usage
\end{itemize}

\subsubsection{Expected Results}
We anticipate this hybrid approach to leverage the strengths of both algorithms. Merge Sort provides a stable O(n log n) performance for larger inputs, while Insertion Sort is highly efficient for small, nearly-sorted subarrays. The in-place merge should help in reducing the space complexity.

\subsubsection{Performance Analysis}
%\begin{figure}[H]
%    \centering
%    \includegraphics[width=0.8\columnwidth]{placeholder_plot2.png}
%    \caption{Performance plot for Hybrid Merge-Insertion Sort}
%    \label{fig:variant2_plot}
%\end{figure}

The performance plot reveals interesting characteristics of our hybrid algorithm:
\begin{itemize}
    \item For small inputs (n < 1000), it significantly outperforms the standard Merge Sort due to the Insertion Sort optimization.
    \item For medium to large inputs, it maintains performance close to or slightly better than standard Merge Sort.
    \item The in-place merge implementation shows a slight performance hit for very large inputs but provides substantial memory savings.
\end{itemize}

\subsubsection{Potential Improvements}
To enhance this variant further, we could:
\begin{itemize}
    \item Fine-tune the threshold for switching between Merge Sort and Insertion Sort
    \item Implement a more efficient in-place merge algorithm to reduce the performance hit for large inputs
    \item Explore parallelization of the merge step for multi-core systems
\end{itemize}
\subsection{Variant 3: }

\subsubsection{Changes Made}
In this variant, we implemented an optimized version of the Quicksort algorithm. The main changes include:
\begin{itemize}
    \item Using a median-of-three method for pivot selection
    \item Implementing an insertion sort for small subarrays (n < 10)
    \item Utilizing tail recursion elimination
\end{itemize}

\subsubsection{Expected Results}
We expect to see improved performance, especially for larger datasets. The median-of-three pivot selection should provide better partitioning, reducing the likelihood of worst-case scenarios. The insertion sort for small subarrays should decrease the overhead of recursion for small inputs, and tail recursion elimination should help in reducing stack usage.

\subsubsection{Performance Analysis}
%\begin{figure}[H]
%    \centering
%    \includegraphics[width=0.8\columnwidth]{placeholder_plot1.png}
%    \caption{Performance plot for Optimized Quicksort}
%    \label{fig:variant1_plot}
%\end{figure}

The performance plot shows a significant improvement in execution time compared to the baseline Quicksort implementation. We observe that:
\begin{itemize}
    \item For small input sizes (n < 1000), the optimized version performs slightly better due to the insertion sort optimization.
    \item For larger inputs, we see a more pronounced improvement, with the optimized version consistently outperforming the baseline.
    \item The performance gain increases with input size, suggesting that our optimizations are particularly effective for larger datasets.
\end{itemize}

\subsubsection{Potential Improvements}
To further enhance this variant, we could:
\begin{itemize}
    \item Experiment with different thresholds for switching to insertion sort
    \item Implement a three-way partitioning scheme to handle duplicate elements more efficiently
    \item Explore cache-friendly memory access patterns to improve performance on modern hardware
\end{itemize}

\subsection{Variant 4:}

\subsubsection{Changes Made}
In this variant, we implemented an optimized version of the Quicksort algorithm. The main changes include:
\begin{itemize}
    \item Using a median-of-three method for pivot selection
    \item Implementing an insertion sort for small subarrays (n < 10)
    \item Utilizing tail recursion elimination
\end{itemize}

\subsubsection{Expected Results}
We expect to see improved performance, especially for larger datasets. The median-of-three pivot selection should provide better partitioning, reducing the likelihood of worst-case scenarios. The insertion sort for small subarrays should decrease the overhead of recursion for small inputs, and tail recursion elimination should help in reducing stack usage.

\subsubsection{Performance Analysis}
%\begin{figure}[H]
%    \centering
%    \includegraphics[width=0.8\columnwidth]{placeholder_plot1.png}
%    \caption{Performance plot for Optimized Quicksort}
%    \label{fig:variant1_plot}
%\end{figure}

The performance plot shows a significant improvement in execution time compared to the baseline Quicksort implementation. We observe that:
\begin{itemize}
    \item For small input sizes (n < 1000), the optimized version performs slightly better due to the insertion sort optimization.
    \item For larger inputs, we see a more pronounced improvement, with the optimized version consistently outperforming the baseline.
    \item The performance gain increases with input size, suggesting that our optimizations are particularly effective for larger datasets.
\end{itemize}

\subsubsection{Potential Improvements}
To further enhance this variant, we could:
\begin{itemize}
    \item Experiment with different thresholds for switching to insertion sort
    \item Implement a three-way partitioning scheme to handle duplicate elements more efficiently
    \item Explore cache-friendly memory access patterns to improve performance on modern hardware
\end{itemize}



\section{Conclusion}
Our refinements demonstrate significant improvements over baseline sorting implementations. The Optimized Quicksort shows excellent performance for large datasets, while the Hybrid Merge-Insertion Sort provides a good balance of performance across different input sizes with reduced memory usage. These optimizations showcase the practical application of concepts from our computer science curriculum, particularly in algorithm analysis, data structures, and low-level system optimizations. Moving forward, we plan to explore more advanced techniques such as cache-conscious algorithms and parallel sorting methods to push the boundaries of sorting performance.

\end{document}